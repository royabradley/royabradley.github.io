\documentclass[]{standalone}

\usepackage[]{tikz}
\usetikzlibrary{calendar,shadows,matrix}

\begin{document}

    \makeatletter

    % This way you can define your own conditions, for example, you
    % could make something as `full moon', `even week', `odd week',
    % et cetera. In principle. The math in TeX could be hard.
    \pgfkeys{/pgf/calendar/start of year/.code={%
        \ifnum\pgfcalendarifdateday=1\relax%
            \ifnum\pgfcalendarifdatemonth=1\relax\pgfcalendarmatchestrue\fi%
        \fi%
    }}%

    % Define our own style
    \tikzstyle{week list sunday}=[
        % Note that we cannot extend from week list,
        % the execute before day scope is cumulative
        execute before day scope={%
               \ifdate{day of month=1}{\ifdate{equals=\pgfcalendarbeginiso}{}{
               % On first of month, except when first date in calendar.
                   \pgfmathsetlength{\pgf@y}{\tikz@lib@cal@month@yshift}%
                   \pgftransformyshift{-\pgf@y}
               }}{}%
        },
        execute at begin day scope={%
            % Because for TikZ Monday is 0 and Sunday is 6,
            % we can't directly use \pgfcalendercurrentweekday,
            % but instead we define \c@pgf@counta (basically) as:
            % (\pgfcalendercurrentweekday + 1) % 7
            \pgfmathsetlength\pgf@x{\tikz@lib@cal@xshift}%
            \ifnum\pgfcalendarcurrentweekday=6
                \c@pgf@counta=0
            \else
                \c@pgf@counta=\pgfcalendarcurrentweekday
                \advance\c@pgf@counta by 1
            \fi
            \pgf@x=\c@pgf@counta\pgf@x
            % Shift to the right position for the day.
            \pgftransformxshift{\pgf@x}
        },
        execute after day scope={
            % Week is done, shift to the next line.
            \ifdate{Saturday}{
                \pgfmathsetlength{\pgf@y}{\tikz@lib@cal@yshift}%
                \pgftransformyshift{-\pgf@y}
            }{}%
        },
        % This should be defined, glancing from the source code.
        tikz@lib@cal@width=7
    ]
    \makeatother

    \tikzstyle{labest}=[font=\footnotesize, fill=orange!30, inner sep=2pt]
    \tikzstyle{evento}=[font=\footnotesize, fill=red!50, inner sep=2pt]

    % The actual calendar is now rather easy:
    \begin{tikzpicture}[every calendar/.style={
            month label above centered,
            month text={\textit{\%mt}},
            %if={(Sunday) [blue!30]},
            %if={(Saturday) [black!30]},
            if={(weekend) [black!30]},
            week list sunday,
        }]
        \matrix[column sep=1em, row sep=1em] {
            \calendar (jun) [dates=2022-06-01 to 2022-06-last]
              if (equals=06-20, equals=06-27) [blue]
              if (between=06-13 and 06-16) [orange]; &
            \calendar (jul) [dates=2022-07-01 to 2022-07-last]
              if (equals=07-11, equals=07-18, equals=07-25) [blue]
              if (equals=07-04) [orange];&
            \calendar[dates=2022-08-01 to 2022-08-last] 
              if (equals=08-01, equals=08-08, equals=08-15) [blue]; \\
        };

        \draw[black] ([xshift=-0.5cm]jun-2022-06-13) |- +(-0.2,1) node [labest,left,align=left] {Summer \\ Sanctus};
        \draw[black] (jul-2022-07-04) |- +(-1,0.5) |- +(-1,-2.25) node [labest,left,align=left,anchor=north] {Independence Day};


        \node[above,yshift=0.5cm,font=\large\bfseries] at (current bounding box.north) {--- Summer 2022 ---};

        \matrix[draw,anchor=north west,xshift=1ex,matrix of nodes,row sep=.5ex,column 2/.style={text width=2.2cm,align=left,anchor=center},xshift=-3.3cm,yshift=0.3cm] 
        at (current bounding box.south east) {
        \node[circle,fill=blue,draw=blue,
        general shadow={fill=gray!60}] {};&[0.15cm] 
        \node[xshift=0cm] {\small{Meeting dates}}; \\
        };

    \end{tikzpicture}
\end{document}